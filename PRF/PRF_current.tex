\documentclass[10pt]{article}
\usepackage[letterpaper,top=1in,bottom=1in,left=1in,right=1in]{geometry}
%\usepackage{geometry}
\usepackage{graphicx}
\usepackage{amssymb}
\usepackage{amsmath}
\usepackage{enumerate}
\usepackage{float}
\setlength\parindent{0pt}

\title{
\vspace{-20.mm}
Privacy Preserving Biometrics based Remote Authentication Protocol for Mobile Devices}

\begin{document}
\maketitle
\section{Motivation:}
There has been a major shift from traditional passwords based authentication to biometrics based authentication in consumer applications in the 
recent past as major service providers such as the leaders in banking~\cite{citi, hsbc, usaa}, credit cards~\cite{mastercard} and 
e-commerce~\cite{amazon, alibaba} are adopting biometrics to authenticate users.
Biometrics is a strong factor of authentication due to its ability to uniquely identify an individual. Different 
vendors have adopted it for different motivations, for examples, Amazon uses selfies based on facial recognition techniques to avoid the difficulty 
in typing passwords to authenticate transactions in mobile devices with small screens~\cite{amazon} and  Master cards has adopted it in order to 
drastically cut-down the cost of false decilned transactions~\cite{mastercard}.

Two main contexts in which biometrics is being used for authentication are: in-person authentication~\cite{google} and remote 
authentication~\cite{hsbc}. In the first case, user is present at the authenticator's premise when authentication is performed and a device of the 
autheticator captures the 
biometrics. In the second case, authentication is performed over the network and the biometrics is captured by the user's device. 
While there are common challenges w.r.t both cases due to the sensitivity (being tightly coupled with one's identity) non-repeatability (no 
two biometrics samples of the same individual match exactly) and non-revocable (inability to cancel/renew) nature of biometrics, the second case 
involves more challenges than the first one in terms of liveliness verification of the biometrics to avoid spoofing attacks, secure 
transmission of the authentication information and security of the user's device. Furthermore, remote biometrics based authentication is being widely 
used since online-banking and e-commerce applications are increasingly adopting it.

Aforementioned commercial authentication systems that are being deplyoed today, inherit key security 
concerns irrespective of the fact that they incorporate state-of-the-art facial/voice recognition algorithms and liveliness verification techniques. 
First, since users' biometrics templates are stored in the server databases for matching during the authentication, they become major targets 
of attackers.
For examples, in Google Hands Free system~\cite{google}, user's picture taken at the authentication time is matched with the user's Hands Free 
profile picture and in Citi bank system, user's pre-recorded voice samples are matched with the voice captured when the user call in. Second, 
multiple third party service providers are storing different biometrics traits of the same user (such as face, voice and fingerprint) for their 
proprietaty authentication protocols. This creates multiple points of vulnerability on one's biometrics identity, due to 
linkability~\cite{linkability}. Third, the current protocols require the users to send a raw biometrics sample over the network each time the user 
remotely authenticates using biometrics, which is not desirable.
%as such biometrics samples reveal sensitive features of the user's biometrics which are used to uniquely identify an individual. 
Stolen biometrics templates from the server databases or from the authentication channels lead to identity theft which poses severe threat to user's 
digital identity, compared to the case in which a password is stolen, because biometrics samples reveal sensitive features of the user's 
biometrics identity which can not be revoked.
%non-revocable nature of the biometrics.

Therefore, it is best to avoid storing and transmitting sensitive biometrics information during authentication, when developing a secure biometrics 
based remote authentication protocol.
%because we can not solely rely on 
%encryption to protect biometrics databases and authentication channels as there have been many instances of 
%password breaches in the past, although such techniques were used to secure passwords during storage and transmission.
The second issue mentioned above can be avoided by getting a trusted identity provider (IDP) to enroll user's biometrics identity~\cite{google, 
identityX}. However, if the IDP is involved in each transaction that the user authenticates, it undermines the user's privacy since the IDP gets to 
know about different transactions that the user performs with different service providers.
  
To address aforementioned security and privacy concerns, we aim to develop a biometrics based remote authentication protocol with following 
characteristics:
\begin{enumerate}
 \item User's biometrics identity is enrolled with only a trusted IDP (i.e: biometrics is not stored with multiple service providers).
 \item Sensitive features of user's biometrics is not stored anywhere.
 \item Sensitive features of user's biometrics is not revealed during authentication.
 \item After the initial enrollment with the IDP, users can carry out biometrics based authentication with multiple service providers without 
involving the IDP. (i.e: user-centric authentication protocol as opposed to traditional IDP-centric authentication protocol).
 \item Efficient in terms of computation and communication in order for it to be carried out from user's mobile devices.
\end{enumerate}

In what follows we describe our roadmap of realizing the goal of developing a privacy preserving biometrics based remote authentication protocol with 
the aforementioned characteristics.

\section{Past and on-going research work:}
We found Zero Knowlede Proof (ZKP) of identity to be a suitable cryptographic primitive to be used along with a secure commitment 
scheme~\cite{pedersenCommitment}, in order to authenticate without revealing any sensitive biometrics information to the SP. The concept of ZKP of 
identity, first proposed by Feige, Fiat and Shamir~\cite{fiat-shamir} in 1988, has been used to develop numerous identity based authentication 
schemes~\cite{idemixConcepts, DAA} for static identities such as email, credit card number, etc. Making it applicable in the domain of biometrics 
based remote authentication is not straight forward due to non-repeatable nature of biometrics identity. 
%omit following sentence if needed%
In other words, since the biometrics sample used to create the commitment does not exactly match the biometrics sample captured during 
authentication, the ZKP might not suceed even for the genuine prover, unlike in the case of static identity.
Previous approaches which addressed the non-repeatability of biometrics, mainly in the domain of biometrics based encryption, have used distance 
matching with threshold, by applying error correction on the features extracted from the second sample. 
Our goal is to generate a unique, repeatable and revocable biometrics based identifier for the user, based on discriminative features of his/her 
biometrics, which is used in creating the identity commitment during the enrollment of identity and which is used later for authentication 
via ZKP of identity usig such commitment. We employ a machine learning based classification model to generate such biometrics identifier (BID).

\subsubsection*{Overview of the solution}

In what follows we provide an overview of the protocol that we have proposed, using aforementioned building blocks. Please refer to our 
paper~\cite{ours} for more details. This protocol involves three main parties namely: IDP, SP and user and also involves two main phases namely: 
enrollment phase and authentication phase.\\

\textbf{Enrollment Phase:}
First, the IDP trains the classification model using the biometrics features extracted from biometrics images. The output is a 
file encapsulating the base classification model that encodes the information required for prediction. 
During enrollment of each user, the base classification model is customized by randomizing the class labels encoded in the 
model. This step is performed as a security measure to avoid compromising the BIDs of the other users of the system, if one user's device is 
compromized.
Features extracted from the biometrics image of the enrolling user is given as input to the customized classification model and the corresponding 
class label is obtained through prediction. This class label is combined with the user provided secret to generate the BID, which is used in creating 
the identity commitment. The identity commitment, which is based on the Pedersend commitment scheme~\cite{pedersenCommitment} takes the form: C (x, r) 
= gxhr mod p, where x = BID and r = user provided secret. Pawwsowrd based key generation is used to derive multiple secrets used in the 
protocol, from a single password provided by the user. The BID generated in this way preserves the uniqueness as it is based on the discriminative 
features of the user's biometris, repeatable based on the prediction accuracy of the classification model and revocable as the user can cancel any 
existing BID and obtain a new one simply by requesting the IDP to issue a new customized classification model, which will output a different class 
label, and changing the password from which the secret used to create the BID is derived.

At the end of the enrollment process, user is issued an identity token (IDT) signed by the IDP and the customized classification model to be used 
during the authentication phase, which are securely stored in the user's device. The IDT contains the commitment, To and From fields indicating the 
identity of the prover and the verifier of the ZKP verification in which the IDT will be used, expiration time and the signature of the IDT vouching 
for such information.

\textbf{Authentication Phase:}
During the authentication phase, the user provides the IDT issued by the trusted IDP, to the SP and carries out ZKP of the biometrics identity.
Authentication scceeds if the user is able to prove the verifier in zero-knowlege, his/her ownership of the biometrics identity 
encoded in the IDT issued by the trusted IDP. In order to carry out the proof, features extracted from a new biometrics image of the user is given as 
input to the customized classification model stored in the user's device and the BID is generated in the same way as in the enrollment phase. This 
BID and the secret derived from the user's password are used to carry out the ZKP of biometric identity.

This protocol that we have proposed in~\cite{outs}, pocess characteristics 1-5 mentioned in Section 1. The research goal of the current semester 
(Spring 2016) is to implement it in the mobile device and evaluate the performance, in order to confirm its compliance with the sixth 
characteristics as well. 


(from our paper)

Currently, we are implementing it in the mobile phone, to test its feasibility interms of performance and security.
Auth app is separate - which is trusted, which is invoked by any third party client app.


(Write about the ZKPK based solution that we have proposed and currently working on in implementing in the mobile device, to achieve the above 
goal.)


We needed to develop a mechanism that creates repeatable, revocable and 
unique BID. We use ML methods as opposed to traditional distance matching (I need to do a benchmarking to prove that ML outperforms distance 
matching).

Unique, non-repeatable, non-revocable.\\


\section{Next research goal:}
In the aforementioned solution that we develop to achive our goal mentioned in Section 1, we assume the support of Trusted Execution Environment 
(TEE) in the user's mobile device to securely store the artifacts obtained from the IDP during the enrollment phase and to execute BID generation 
using the classifier during the authentication phase. TEE isolates the storage and execution from other applications which provides further 
protection in addition to encrypting the artifacts during storage and which is being used by Apple touch ID and android fingerprint authentication 
framework as well.
While there are extensive research on the develpment of TEE technology~\cite{tee}, there are attacks being discovered~\cite{blackhat} on them too. 
Since widely used implementations of TEE technology such as ARM's TrustZone are proprietary and not disclosed for public review, the level of 
assurance provided against a given threat model is unclear~\cite{armtrustzone}. 

Furthermore, recent advancement in memory forensics techniques~\cite{dimsum}, pose threats on processing sensitive data in the memory of user's 
device in clear text, as it is proved that such data could be recovered from memory images even long after the processing of such data is finished. 
Although it is theoretically possible to clear the memory soon after the processing of sensitive data, it is not practical because it requires 
accurate tracking of each data object in order to clear them upon destruction of the process, which is a very heavy-weight operation 
that no commodity operating system performs.

Therefore, it is at our best interest to develop a privacy preserving biometrics based authentication protocol which achieves all the security and 
privacy goals mentioned in Section 1, based on theoretical foundations and without assuming any hardware security support. Fully homomorphic 
encryption (FHE)~\cite{fhe} is one solution to avoid decryption of sensitive biometrics information during authentication, however, it has certain 
drawbacks to be used in building an authentication protocol with the aforementioned characteristics, as the service provider learns the 
authentication function, which is the classification model in our case, and it needs to obtain the secret key to decrypt the authentication result, 
which in turn allows it to decrypt the biometrics data as well. Yao's garbled circuit~\cite{yaogc}, on the other hand, is a better cryptographic 
building block as it allows secure computation of a function $f$ on input data $x$ to obtain the result $f(x)$ while hiding both the function and the 
input data from the function evaluator.

\subsubsection*{Related work:}
What people have already done in this direction:
- GC for biometrics authentication
- GC on mobile devices
- GC for classification
- Reusabile GC.

\subsubsection*{Proposed Solution:}

\subsubsection*{Challenges:}
One basic limitation of the original garbled circuit construction is that it offers only one-time usage. Specifically, evaluating a circuit on any 
new input requires an entirely new garbling of the circuit~\cite{reusablegc}. This causes the user to communicate with the IDP each time the user 
needs to authenticate to a service provider, which we need to avoid as per the 4th requirement mentioned in Section 1. The problem of reusing garbled 
circuits has been open for 30 years until the reusable garbled circuit contstruct was proposed by Shafi et. al in 2013~\cite{reusablegc}. This new 
construct looks promising to develop the biometrics based authentication protocol that we target, as it allows the secure authentication artifacts to 
be used multiple times without exposing any sensitive information during the process.
Utilizing this construct in designing and developing our target biometrics authentication protocol, however, requires addressing several challenges. 
Firstly, the scheme poposed in~\cite{reusablegc} is presented in the context of two party secure computation. Although original garbled circuit 
construct can be used in three-party context, it is currently not straight forward to see how the re-usable garbled circuit contruct could be 
extended to three-party protocol, which we need to investigate further. Secondly, all the current efficient implementations of garbled circuits do not 
support this construct, which is currently a challenge when it comes to implemention of the protocol. Therefore, we believe that designing and 
developing a privacy preserving biometrics authentication protocol using this construct will contribute important results to the research literature. 

\subsubsection*{Contribution of this work:}
We aim to produce the following research output:
\begin{enumerate}
 \item Designing a privacy preserving biometrics based remote authentication protocol with aforementioned characteristics, using re-usable garbled 
circuit construct. 
 \item A prototype implementation of the protocol that is able to run in mobile devices.
 \item Security and performance analysis of the protocol.
\end{enumerate}


\section{Background and potential impact:}
- After the current work is over, at the end of this semester, we will have a working prototype of our past approach in the mobile phone. PhD student 
Hasini will be attending a summer school conducted by leaders of secure mpc and oblivious computation research to learn the current-state-of-the-art 
techniques. She has got a travel grant form NSF to participate in this.
- Google ATAP project has accepted a proposal written by us and has funded Purdue.
- Can help make existing biometrics based remote authentication in day-to-day transactions such as online-banking and e-commerce secure, which 
already has a large user base and increasing.\\
- Generalization of this techniques can have impact on other ML tasks where privacy is a concern.

\bibliographystyle{IEEEtran}
\bibliography{IEEEabrv,IEEEexample}
 
\end{document}
