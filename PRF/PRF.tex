\documentclass[11pt]{article}
\usepackage{geometry}
\usepackage{graphicx}
\usepackage{amssymb}
\usepackage{amsmath}
\usepackage{enumerate}
\usepackage{float}
\title{
%\vspace{-20.mm}
Privacy Preserving Biometrics based Authentication Protocol for Mobile Devices}


\begin{document}
\maketitle
\section{Motivation:}
Biometrics has been used as a strong factor of authentication. \\
Unique, non-repeatable, non-revocable.\\
Major banks, credit card companies and e-commerce giants are shifting to biometrics based authentication.\\
Different merchants are adapting it for different motivations/goals.\\
In in-person (Google pilot project for in-store profile picture for payment verification) and remote authentication (Amazon and Ali-play). More 
challenges in remore authentication compared to in-person.\\
Popular e-commerce applications moving towards this, for user friendliness (hard to enter passwords in small screens and people forget) and security. 
Although their FRT and liveliness checks is improving, security wise their using traditional template matching. It has several critical risks:\\
- Several SPs maintaining it (e.g: Master card servers, Ali-pay etc.) and worse, different SPs are using different biometric traits of yours. And if 
an attacker get access to your multiple accounts, whole of your biometric identity can be lost.\\
This can be avoided by having a trusted IDP doing the auth verification for you. E.g: Google's Hands Free pay. But then the problem is: IDP knows 
about every place you make purchases which undermines your privacy. While we encourage enrolling biometrics at one trusted party, we do not 
encourage that trusted party being involved in every transaction of the user.\\
- Need to protect the database.\\
- Need to reveal the biometrics each time. Higher risk of getting stolen.\\
- If compromised, higher risk than password.\\
Our goal is to develop a privacy preserving one - no storage of biometrics, no revealation of biometrics during auth, user-centric (no IDP involved 
after initial enrollment.) still the authenticator is confirmed that user is authenticated using his/her biometrics.\\
\section{Past and on-going research work:}
ZKP. This has been used for static identity based authentication for a long time, since its first inception in 1986 - FFS. 
Fulfils all requirements except for one major challenge.
Challenge is using biometrics as the identity due to its non-static nature. We needed to develop a mechanism that creates repeatable, revocable and 
unique BID.
\subsubsection{Overview of the solution}
(from our paper)

Currently, we are implementing it in the mobile phone, to test its feasibility interms of performance and security.
Auth app is separate - which is trusted, which is invoked by any third party client app.
\section{Next research goal:}
In the previous approach, 
- depend upon trust of hardware of phone to protect artifacts stored on the phone. Apple touch ID and android fingerprint authn framework use that. 
But there has been attacks on trust zone uncovered in the past, which has been fixed. Although there is research going on that 
area as well, it is good to rely on them as less as possible, as the tech is not public for review.
- We argue that attack window is very small. But there are recent attacks to recover sensitive info. from physical memory images. Trust zone is just 
virtual world in same physical processor and therefore, should be vulnerable to this, although is not specifically tested by the researchers.

To address such concerns, GC is a good candidate. If we can encode ML model as a function, rather than a file (as we are currently using.)
\subsubsection*{Challenges:}
Reusability of the garbled circuits.
Performance. Protocol achieving all the security and privacy goals.
How to incorporate liveliness check. Have to rely on trusted auth app.
\subsubsection*{Related work:}
What people have already done in this direction:

\subsubsection*{Contribution of this work:}

\section{Background and potential impact:}
- After the current work is over, at the end of this semester, we will have a working prototype of our past approach in the mobile phone. PhD student 
Hasini will be attending a summer school conducted by leaders of secure mpc and oblivious computation research to learn the current-state-of-the-art 
techniques. She has got a travel grant form NSF to participate in this.
- Google ATAP project has accepted a proposal written by us and has funded Purdue.
- Can help make existing biometrics based remote authentication in day-to-day transactions such as online-banking and e-commerce secure, which 
already has a large user base and increasing.\\
- Generalization of this techniques can have impact on other ML tasks where privacy is a concern.


 
\end{document}
