\documentclass[10pt,letterpaper]{article}
\usepackage{multicol} % Used for the two-column layout of the document
\pdfpagewidth=8.5truein
\pdfpageheight=11truein
\usepackage[letterpaper,top=1in,bottom=1in,left=1in,right=1in]{geometry}
\usepackage{amssymb}
\usepackage{amsmath}
\usepackage{mathtools}
\usepackage{graphicx}
\usepackage{float}
\usepackage{times}
\usepackage{url}
\usepackage{hyperref}

%Include whatever packages/commands you need

%Project Title
\title{Privacy Preserving Personalized Medicine Models}
%Your Name
\author{Hasini Gunasinghe}
\date{12/18/2015} 


\begin{document}
\maketitle

\begin{abstract}
With advancement of whole genome sequencing, personalized medicine procedures have been improved to diognose genetic diseases and predict treatment dosages 
according to one's genetic makeup. Despite its significant impact on the advancement of the medical field, these personalized medicine procedures come 
with privacy concerns which can not be taken lightly. Privacy preserving personalized medicine protocols that have been proposed previously suffer from
several common issues. The goal of this project was to design a new privacy preserving personalized medicine protocol using existing privacy enhancing 
techniques as building blocks, in order to address such issues and to evaluate its security. I propose two protocols using garbling scheme and oblivious 
transfer as building blocks of the protocol design. Then I propose a concrete instantiation of the desinged protocols, using specific existing instantiations 
of a garbling scheme and a oblivious transfer protocol, to achieve the set of identified security properties in the personalized medicine protocol 
settings. I plan to implement the proposed protocols and evaluate their performance as future work.
\end{abstract}

\begin{multicols}{2} % Two-column layout throughout the main article text 

%Include your sections here. Introduction and Conclusion are necessary.
\section{Introduction}
Personalized medicine is an emerging area in which a patient's susceptibility to certain diseases is tested as well as treatment dosages are determined 
based on the patient's genetic information. This has led to advancement in diagnosis, preventive medicine and medical treatments in general. 

There are certain privacys concerns associated with such schemes. Firstly, the genetic information is unique to an individual which not only reveals various 
information (such as ethnic heritage, phenotypes, potential susceptibility to genetic diseases etc.) about the individual him/herself, but also reveals such
information about his/her relatives. As of present, the key issue in genetic information of an individual being in the wrong hands is that it can lead to 
discrimination in employment and health insurance. Therefore, patients might want to conduct personalized medicine procedures while
protecting their genome privacy. Although there exits policies such as Genetic Information Non-discrimination Act (GINA), policies alone 
can not enforce genome privacy without proper technical solutions in place. 

Secondly, the personalized medicine models are the outcome of extensive research carried out for many years and they might become intellectual property of 
certain medical services such as pharmaceutical companies, which might want to offer the service of personalized medicine procedures without revealing 
information about the models.

We can identify two main settings in which the personalized medicine procedures are carried out, as illustrated in figure~\ref{prot1} and figure~\ref{prot2}.
In setting 1, the patient himself/herself directly consumes the service of the medical service provider (MSP). In setting 2, the patient performs it at a 
health care provider (HCP) (such as a physician or hospital laboratory), who consumes the service of the MSP.
\begin{figure}[H]
\label{prot1}
\centering
\includegraphics[height=2in,width=2.5in]{setting1}
\caption{Personalized Medicine Protocol Setting 1}
\end{figure}

\begin{figure}[H]
\label{prot2}
\centering
\includegraphics[height=2in,width=2.5in]{setting2}
\caption{Personalized Medicine Protocol Setting 2}
\end{figure}

Key goals in a privacy preserving solution for such personalized medicine procedures are:
\begin{enumerate}
 \item Patient's genetic information is not revealed either to the HCP or to MSP.
 \item Personalized medicine model is not revealed either to the HCP or to the patient.
 \item Result of the test is only known to the patient.
 \item HCP can not produce a valid looking fake result.
\end{enumerate}

\section{Related Work}
Current state-of-the-art solutions for privacy preserving personalized medicine methods are found in~\cite{Ayday, Cristofaro, Djatmiko}. The work by 
Ayday et. al~\cite{Ayday} has laid the foundation on top of which the latter two parallel works are built. In \cite{Ayday}, authors have proposed a privacy 
preserving solution for SNP (single nucleotide polymorphisms)~\cite{SNP} based disease risk susceptibility tests which improves storage efficiency over 
having to analyze the whole genome. The authors have used a variation of Pailler additive homomorphic encryption~\cite{BCP} and proxy re-encryption to 
achive the privacy goals mentoned in Section 1. 

The authors of~\cite{Cristofaro} point out several limitations of the previous work such as high storage and computation complexity due to the
way the SNPs are represented, high computation complexity due to the choice of the encryption scheme and revealing of which and how many SNPs are used.
Going further, they have introduced a new efficient encoding of SNPs and have utilized additively homomorphic elleptic curve based ElGamal cryptosystem in
their approach which have helped to overcome the drawbacks they have identified in the previous work. Both of the above work have presented a general
solution without considering any specific case study. 

On the other hand, the authors of~\cite{Djatmiko} have built a concrete solution for privacy preserving evaluation of the popular pharmacogenomic model for 
personalized Warfarin dosing~\cite{Warfarin} based on the work by Ayday et. al~\cite{Ayday}. They have utilized private information retrieval to help the 
medical unit to select the patient's information to be used in the model evaluation in a privacy preserving manner. However, their work also suffer from 
the issue of revealing how  many attributes are involved in the model and in addition, the patient learns which non-genomic attributes are used in the model.

In general, I identified that all the aforementioned previous work suffer from three main common issues:
\begin{enumerate}
 \item They need to store the state of all SNP positions of human genome which is around 40 million where as an individual only carries 4 million SNPs 
 on average. 
 The reasons for above works to store the states all SNP positions is to make sure that actual SNP positions of an individual are not revealed during 
 protocol execution, which is not storage efficient. 
 \item They can be applied only to linear models. However, there can be personalized medicine models which are not linear, as they are based on 
 machine learning models.
 \item They involve several interactions back and forth in order to carry out the protocol.
 \item Each of them address either setting 1 or setting 2, not both.
\end{enumerate}

\section{Preliminaries}
In this section, I introduce the notation used in the protocol designs. Particularly, figure~\ref{gc}, illustrates the key components of a garbling scheme.
\begin{figure}[H]
\label{gc}
\centering
\includegraphics[height=1.5in,width=3.0in]{garbled_circuit}
\caption{Key components of a garbling scheme}
\end{figure}
 
\section{Abstract Solution Design}
In this section, I present the designs of the two privacy preserving personalized medicine protocols for the two different protocol settings discussed in 
Section 1. 
\begin{figure*}
\label{prot11}
\centering
\includegraphics[height=3in,width=\textwidth]{prot1}
\caption{Design of privacy preserving personalized medicine protocol 1, for the personalized medicine prototocl setting shown in ~\ref{prot1}}
\end{figure*}

\begin{figure*}
\label{prot22}
\centering
\includegraphics[height=3in,width=\textwidth]{prot2}
\caption{Design of privacy preserving personalized medicine protocol 1, for the personalized medicine prototocl setting shown in ~\ref{prot1}}
\end{figure*}

Since we have the requirement of protecting the privacy of both the inputs from the patient and the function of the MSP, above two protocols are 
inherited from the private function evaluation (PFE) protocol design in general. 

\section{Concrete Solution Design}
In the actual implementation of the above protocol design, the garbling scheme will be implemented with the dual-key cipher based Garble-2 scheme presented 
in ~\cite{FoundationsGC} and the oblivious transfer protocol will be implemented with the one presented in~\cite{OT}.
The reasons for using the above schemes:
\begin{enumerate}
 \item Garble 2 is the only garbling scheme that achieves all three properties of: privacy, obliviousness and authenticity. Please refer to the proof of 
 those properties of Garble-2 in~\cite{FoundationsGC}.
 \item The oblivious transfer protocol presented in~\cite{OT} is a simple and efficient protocol with a publicly available implementation.
\end{enumerate}

\subsection{Why not using an existing garble circuit framework?}
I looked into the garble circuit framework provided in ObliVM~\cite{ObliVM}, to see if it can be directly used to implement the solution, rather than 
designing and implementing a protocol using separate building blocks. However, it seemed that the ObliVM is focused on the support of garbling scheme
for secure function evaluation (SFE) (in which the function is assumed to be private and only the privacy of the inputs and outputs need to be protected.)
where as our requirement is using garbling scheme for private function evaluation (PFE). As mentioned above, Garble-2 in ~\cite{FoundationsGC} is the
only garbling scheme that I found to satisfy all three secrity properties that are required. However, Garble-2 is only provides a garbling scheme and
does not come with an oblivious transfer protocol. Hence, I selected the aforementioned instance of oblivious transfer for the concrete instantiation
of the two protocols.


\section{Analysis}

The concrete instantiation of the two proposed protocols satisfy privacy, obliviousness and authenticity properties which are inherited from the
building block of Garble-2 garbling scheme. The definition of these three security properties and which requirements identified in the section 1 are 
addressed by each of them, is mentioned below:
\begin{enumerate}
 \item Privacy: A party acquiring $<F, X, d>$ should not learn anything beyond the final output $y$ and what is revealed from the side information 
 function. This addresses requirements 1 and 3 identified in section 1. 
 \item Obliviousness: A party acquiring $<F, X>$, but not $d$, should not learn anything about $f, x$ and $y$ beyond what is revealed from the side information 
 function. This addresses requirement 2 identified in section 1.
 \item Authenticity: A party acquiring $<F, X>$ should not be able to produce $Y'$ different from $F(X)$ that seems to be valid. 
 This addresses requirement 4 identified in section 1.
\end{enumerate} 

Advantages of the proposed protocols over previous work are: they present a generic solution that address requirements of both the protocol settings 
discussed in section 1, they can be applied for non-linear models and the user does not have to store all SNP positions in these protocols.

Drawbacks of the proposed protocols can be viewed as: oblivious transfer step being a bottleneck as a communication overhead, circuit size 
(number of truth tables to be transferred) being a communication overhead and the user learning the function parameters.


\section{Conclusion}
In this project, I selected genome privacy as the area to explore, which was a new and exciting topic to me. After reading several survey papers in genome
privacy, I opted to investigate on privacy of personalized medicine protocols as I noticed that there are several common issues in the existing privacy
preserving solutions which are built based on homomorphic encryption schemes. I decided to design a solution based on garbling schemes which was also 
new to me. I made that choice, because they cater security requirements of private function evalution (PFE) and they are identified to be among the fastest 
secure multi-party computation techniques in the literature~\cite{ObliVM}. The protocols that I have designed achieves privacy, obliviousness and authenticity with the 
specific instantiations of the garbling scheme and oblivious transfer protocol that I plan to use. 
Poposed protocols address the issues identified in the previous solutions. I have provided an informal analysis of the protocols. As future work, I plan to 
formally analyze the protocols and evaluate the performance of a real implementation of these proposed protocols.

{\raggedright
\small
\bibliographystyle{IEEEtran}
\bibliography{IEEEabrv,IEEEexample}
}
\appendix
%If you want to include something in appendix
\end{multicols}	
\end{document}
