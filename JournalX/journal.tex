\documentclass[11pt]{article}
\usepackage{geometry}
\usepackage{graphicx}
\usepackage{amssymb}
\usepackage{amsmath}
\usepackage{enumerate}
\usepackage{float}
\setlength\parindent{0pt}

\title{
\vspace{-20.mm}
My Journal on My Research Work}
\author{Hasini Gunasinghe (huralali@purdue.edu)}
\date{}

\begin{document}
\maketitle

\section*{RahasNym}
Until 12th Feb, I mostly worked on improving the paper to be submitted to SACMAT.

\pagebreak

\section*{Biometrics Research Work}
\subsection*{Android Development:}
\subsubsection*{Wed Jan 20}
I installed Android studio. I referenced this good tutorial: http://www.androidauthority.com/first-android-app-what-you-need-to-know-619260/\\
However, in my Ubuntu, I got an error from android studio saying that SDK or libraries could not be installed. \\
Then this was the solution: http://stackoverflow.com/questions/28804863/android-studio-how-to-install-android-platform-tools-on-ubuntu-14-04-64-bit
It was because Android needs 32 bit libs and I have a 64 bit Ubuntu.\\
Here are some tips I found to make the emulator fast:
http://developer.android.com/tools/devices/emulator.html\#vm-linux

\subsubsection*{Feb 19}
Emulator runs Android in a kind of virtual machine, as an Android phone with an Intel processor. This is faster than emulating an ARM processor on 
your PC.

\subsubsection*{Feb 22}
After lunch, I worked on the mobile app dev. I am still at the very very beginning. Followed first app tutorial till end, and got a problem when 
running in the emulator. Emulator needs KVM \textit{emulator: ERROR: x86 emulation currently requires hardware acceleration!
Please ensure KVM is properly installed and usable.
CPU acceleration status: KVM is not installed on this machine (/dev/kvm is missing).} \\

Then I tried to install KVM based on this tutorial:\\
https://software.intel.com/blogs/2012/03/12/how-to-start-intel-hardware-assisted-virtualization-hypervisor-on-linux-to-speed-up-intel-android-x86-emul
ator

However, there seems to be problems. \\
1. When I ran the command at the beginning of that tutorial to check if the CPU supports KVM extensions, I get 
the output as NO. However, since the error from Android studio shows some hope, I tried to install it.

2. Then the install command given in the tutorial doesn't work. Then I tried this: whose command works. 
https://www.howtoforge.com/tutorial/kvm-on-ubuntu-14.04/

It seems that now I need to relogin to enable KVM for my user accounts. 

\subsubsection*{Feb 24:}
Since I had issues in running the hello world app in the emulator due to KVM enabling issue, I thought f checking my bios to see if KVM is enabled.\\
Before accessing the bios, I needed to backup my important files in bitbucket. Therefore, I spent sometime backing up my files.\\
Then I spent some fair amount time accessing the bios. First I tried F2 key as mentioned in many online articles which didn't work. Then F12 worked. 
Intel VT is enabled in the bios.\\
Then I checked the VMWare settings. So the problem was VT was not enabled in VMWare. When I enabled it, the emulator ran. But it was damn slow.\\

I searched about how to make the android studio fast. I got some answers:\\
1. http://www.viralandroid.com/2015/08/how-to-make-android-studio-fast.html\\
This suggested either to test in real device or in genymotion emulator: https://www.genymotion.com/download/\\

\textbf{Option 1: } I tried to install genymotion emulator, but it requires VirtualBox to run. Hmmmm. VirtualBox on VMWare!. So I thought of 
first running in the real device and see because I anyway have to install it in a real device.

\textbf{Option 2: } Option 2 is to install genymotion+virtual box on windows, compile the project in Linux and run it in the genymotion 
running in Windows. https://dzone.com/articles/genymotion-simply-best-android \\

I tried to install it in my phone (I upgraded my phone too for this reason. My phone is running Android 4.4.2). But somehow it didn't get installed.
I was so tired by then and went to sleep.\\

\textbf{TODO:}
Read how to make android studio faster:\\
1. https://dzone.com/articles/how-speed-android-studio\\
2. http://www.codeproject.com/Articles/803935/How-To-Make-Android-Studio-Really-Fast-On-A-Window\\


\subsubsection*{Feb 25:}
I continued my attempt in installing in the real device. But I still couldn't.
I found that there is no apk built when I run the project. So the studio complains that no local path exists. So I need to understand what the heck 
is going through the Android build process. See how slow my progress is! :(\\

Oh... Android studio integrated gradle build is a crap!. It didn't even complain that the build was not successful. That is why there was no apk 
local path existed.\\

I ran gradle build from command line according to this tutorial: http://developer.android.com/training/basics/firstapp/running-app.html\\
Steps: \\
1. invoke the assembleDebug build task using the Gradle wrapper script (gradlew assembleRelease).\\
chmod +x gradlew\\
./gradlew assembleDebug\\

I got an error when running ./gradlew: ``libz.so.1: cannot open shared object file'', followed by: ``Exception in thread "png-cruncher\_4" 
java.lang.RuntimeException: Timed out while waiting for slave aapt process, try setting environment variable SLAVE\_AAPT\_TIMEOUT to a value bigger 
than 5 seconds''.\\
It seemed that both errors are due to not installing the library: zlib1g as mentioned in: 
http://stackoverflow.com/questions/21256866/libz-so-1-cannot-open-shared-object-file\\
After doing: sudo apt-get install zlib1g:i386, the gradle gave me: BUILD SUCCESSFUL Total time: 1 mins 16.537 secs. :)\\

2. Then I wanted to deploy the app in my phone also from the command line (mentioned in the same above tutorial).\\
Added android-sdk/platform-tools to PATH and ran: adb install app/build/outputs/apk/app-debug.apk\\
Yeyyy... I got my first android app installed on my phone! :) Now the floor is mine to do cool stuff.\\

Command line worked so smooth compared to crappy android studio. \\

I went step further, I editted my app and tried to run again in my app. First I failed because it was already installed. Solution was to install it 
with the -r option. :)

\pagebreak
\subsection*{To Dos:}
\begin{enumerate}
 \item Study the mobile phone's security architecture and find about the latest work in the secure architecture/trust zone.
 \item Discuss the attack surface/vulnerability window and argue that it is very small.
 \item Look at Daniel's journal paper to see how to prepare a journal paper.
\end{enumerate}

\subsection*{Literature Survey:}
\subsubsection*{Feb 19}
Today, I was just searching zero knowledge biometrics authentication for remote services. I got a bunch of results - papers and a commercial
product.\\
\textbf{Sedicii}\\
This commercial product : sedicii (https://www.sedicii.com), seem to be doing exactly what I have done: ZKP based identity verification/authorization.
They say that they do credit card authorization as well as biometrics authorization in ZKP - exactly my two works.
They have not described how they do biometrics authentication in ZKP, however, it should be similar to their website loging scenario:
I have written how their credit card authorization is comparable to ours in my RahasNym journal.

\subsection*{Brain Storming:}
\subsubsection*{Feb 19}
Lot of biometrics based authentication mechanisms are defined for authenticating to devices. Once authenticated into the device, different services 
that the user accesses are already logged in with username/password security. In such cases, critical remote services are relying on the device 
biometric authentication, which is not usually strong.\\
\textbf{TODO:} \\ See how device biometrics authentication works in Android and Apple.\\
Also, if a malware is installed by some mistake by the user, client of the remore service is at risk (password can be stolen, session stolen etv.).\\ 
\textbf{TODO:}\\
See how bank apps work in mobile devices.\\

This shows the requirement for remote services to have their own authentication of user to make sure that the genuine user invokes some request, with 
strong verification, beyond usernam/password, and without relying on device autentication.\\
ZKP is a good candidate. There are some previous works, suffers from some drawbacks.\\ Main issue is identity is not static.\\
The works differ by the approach they address this non-static nature of the biometrics.\\

This should be a standard mechanism, that any app resides in user's device-communicating with the remote service can integrate easiliy.\\
\textbf{TODO:}\\
See this could be developed as a service in Android which could be invoked by other apps.\\

\textbf{Contributions of our work:}
\begin{itemize}
 \item Secure protocol for remote authentication using biometrics. Preserves good properties of biometrics (i.e: uniqueness). Avoids non-desirable 
properties of biomertics (i.e: non-repeatability, non-revocability).
 \item Prototype implementation that is a proof of concept. That can be integrated to any app.
 \item Security Analysis and Performance Analysis.
\end{itemize}

\subsubsection*{Feb 20}
I started documenting what I wrote down on paper during the weekend. I felt I need a brainstorming/mind mapping tool. And I got FreeMind and noted 
down different aspects.\\


\pagebreak
\section*{New Ideas:}
\subsection*{PrivBioAuh and PrivBioGeneAuth}
I can have two projects under the umbrella: PrivBioAuth. Two will be: PrivBioMTAuth and PrivBioGeneAuth.\\

\subsection*{How to address the security issues in health care data collected from IoT devices using biometrics.}
Professor suggests to see how to generate keys from biometrics to protect the medical data.

\subsection*{Fuzzy Zero Knowledge Scheme:}
\subsubsection*{Feb 23:}
I worked on the research the whole afternoon after lunch - but on an apparently useless one. I tried to comeup with a fuzzy zero-knowledge protocol 
based on the ideas from fuzzy identity based encryption. But I had no luck.

I need to look at fuzzy commitment and fuzzy vault before talking about this with anyone else.\\
Professor mentioned we can talk with Professor Atallah or Italian professors.\\

\subsection*{Storing the machine learning model by encoding it in a garbled circuit}
\subsubsection*{Feb 19}
- I got an idea yesterday that I could use the same method of obtaining a circuit from IDP for authentication verification later.\\ 
But the circuits can be used only once, which is a problem.\\
\textbf{QUESTION:}\\
- Good question I can ask in the summer school: are garbled circuits re-usable?\\

\textbf{REFERENCES:}\\
This is a very nice project website - a resovair of resources. http://www.mightbeevil.org/ and http://www.mightbeevil.org/mobile/


\pagebreak
\section*{Meetings with the advisor:}
\subsection*{Feb 25:}
\begin{itemize}
 \item I spent lot of time getting ready for the meeting - how I explain what I did, and my ideas to the Professor. At the beginning, she was not in 
a very good mood. But the end was very good.
\item It is ok to have 35\% overlap with the conference paper. If I could get the implementation on the mobile phone and do the Performance 
evaluation, that would be an excellent extension for the journal paper. Need to get it by the end of this semester. Usually we do not submit new work 
for journals. Only extensions.
\item I need to look at Daniel's journal to see how I can prepare my journal paper.
\item I need to study the mobile phone android security architecture and write about it.
\item Professor mentioned about the new idea of protecting the medical data.
\item Professor was interested in two of my new ideas. Professor mentioned we can ask Prof. Atallah or her Italian collaborators. She said it is a 
very specialized area.
\item Professor gave me the two recommendations for the communication requirement and the travel grant application.

\end{itemize}

\end{document}