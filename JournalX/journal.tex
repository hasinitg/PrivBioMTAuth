\documentclass[11pt]{article}
\usepackage{geometry}
\usepackage{graphicx}
\usepackage{amssymb}
\usepackage{amsmath}
\usepackage{enumerate}
\usepackage{float}
\setlength\parindent{0pt}

\title{
\vspace{-20.mm}
My Journal for Biometrics Work}
\author{Hasini Gunasinghe (huralali@purdue.edu)}
\date{}

\begin{document}
\maketitle

\section*{Wed Jan 20}
I installed Android studio. I referenced this good tutorial: http://www.androidauthority.com/first-android-app-what-you-need-to-know-619260/
However, in my Ubuntu, I got an error from android studio saying that SDK or libraries could not be installed. 
Then this was the solution: http://stackoverflow.com/questions/28804863/android-studio-how-to-install-android-platform-tools-on-ubuntu-14-04-64-bit
It was because Android needs 32 bit libs and I have a 64 bit Ubuntu.
Here are some tips I found to make the emulator fast:
http://developer.android.com/tools/devices/emulator.html\#vm-linux

\section*{Feb 19}
Today I am back to my biometrics research work. I was away coz I worked on the SACMAT paper, was sick and was catching up with Programming
Languages course.\\
\textbf{IDEA:}
- I got an idea yesterday that I could use the same method of obtaining a circuit from IDP for authentication verification later.\\ 
But the circuits can be used only once, which is a problem.\\
\textbf{QUESTION:}\\
- Good question I can ask in the summer school: are garbled circuits re-usable?

\subsection*{Android Dev:}
Emulator runs Android in a kind of virtual machine, as an Android phone with an Intel processor. This is faster than emulating an ARM processor on 
your PC.

\subsection*{Literature Survey:}
Today, I was just searching zero knowledge biometrics authentication for remote services. I got a bunch of results - papers and a commercial
product.\\
\textbf{Sedicii}\\
This commercial product : sedicii (https://www.sedicii.com), seem to be doing exactly what I have done: ZKP based identity verification/authorization.
They say that they do credit card authorization as well as biometrics authorization in ZKP - exactly my two works.
They have not described how they do biometrics authentication in ZKP, however, it should be similar to their website loging scenario:
I have written how their credit card authorization is comparable to ours in my RahasNym journal.

\subsection*{Brain Storming:}
Lot of biometrics based authentication mechanisms are defined for authenticating to devices. Once authenticated into the device, different services 
that the user accesses are already logged in with username/password security. In such cases, critical remote services are relying on the device 
biometric authentication, which is not usually strong.\\
\textbf{TODO:} \\ See how device biometrics authentication works in Android and Apple.\\
Also, if a malware is installed by some mistake by the user, client of the remore service is at risk (password can be stolen, session stolen etv.).\\ 
\textbf{TODO:}\\
See how bank apps work in mobile devices.\\

This shows the requirement for remote services to have their own authentication of user to make sure that the genuine user invokes some request, with 
strong verification, beyond usernam/password, and without relying on device autentication.\\

This should be a standard mechanism, that any app resides in user's device-communicating with the remote service can integrate easiliy.\\
\textbf{TODO:}\\
See this could be developed as a service in Android which could be invoked by other apps.\\

\textbf{IDEA:}\\
I can have two projects under the umbrella: PrivBioAuth. Two will be: PrivBioMTAuth and PrivBioGeneAuth.\\

\textbf{Contributions of our work:}
\begin{itemize}
 \item Secure protocol for remote authentication using biometrics. Preserves good properties of biometrics (i.e: uniqueness). Avoids non-desirable 
properties of biomertics (i.e: non-repeatability, non-revocability).
 \item Prototype implementation that is a proof of concept. That can be integrated to any app.
 \item Security Analysis and Performance Analysis.
\end{itemize}


\section*{Feb 22:}
I started documenting what I wrote down on paper during the weekend. I felt I need a brainstorming/mind mapping tool. And I got FreeMind and noted 
down different aspects.\\

After lunch, I worked on the mobile app dev. I am still at the very very beginning. Followed first app tutorial till end, and got a problem when 
running in the emulator. Emulator needs KVM \textit{emulator: ERROR: x86 emulation currently requires hardware acceleration!
Please ensure KVM is properly installed and usable.
CPU acceleration status: KVM is not installed on this machine (/dev/kvm is missing).} 

Then I tried to install KVM based on this tutorial:\\
https://software.intel.com/blogs/2012/03/12/how-to-start-intel-hardware-assisted-virtualization-hypervisor-on-linux-to-speed-up-intel-android-x86-emul
ator

However, there seems to be problems. \\
1. When I ran the command at the beginning of that tutorial to check if the CPU supports KVM extensions, I get 
the output as NO. However, since the error from Android studio shows some hope, I tried to install it.

2. Then the install command given in the tutorial doesn't work. Then I tried this: whose command works. 
https://www.howtoforge.com/tutorial/kvm-on-ubuntu-14.04/

It seems that now I need to relogin to enable KVM for my user accounts. 
\end{document}