\documentclass[11pt]{article}
\usepackage{geometry}
\usepackage{graphicx}
\usepackage{amssymb}
\usepackage{amsmath}
\usepackage{enumerate}
\usepackage{float}
\setlength\parindent{0pt}

\title{
\vspace{-20.mm}
CS57900: Homework 6}
\author{Hasini Gunasinghe (huralali@purdue.edu)}
\date{}

\begin{document}
\maketitle

\section*{Q1}
I have implemented the solution for circular string linearization problem using the suf\_tree.py library and uploaded the code.

The logic is: building the suffix tree on the string obtained by repeating the original string twice and traversing the tree in breadth first manner,
selecting the branch to explore in each iteration that has the smallest starting character in lexiographical order, until the path length of the current node
equals or greater than the length of the original string.

This can be easily done using the suffix array built using the suffix tree, as the suffix array stores the suffixes in lexiographically sorted order.
Following is the code snippet of the algorithm:

\begin{figure}[H]
\centering
\includegraphics[height=3.50in,width=6.00in]{code}
\caption{Code snippet of the algorithm}
\label{example}
\end{figure}


\pagebreak

\section*{Q2}

\pagebreak
\section*{References}

\end{document}